\documentclass[tocnopagenum]{thesis-ekf}
%a4paper, 12pt, 1.5-es sortávolság, margók
\usepackage[T1]{fontenc}
\PassOptionsToPackage{defaults=hu-min}{magyar.ldf}
\usepackage[magyar]{babel}
\usepackage{mathtools,amssymb,amsthm,pdfpages}
\footnotestyle{rule=fourth}
\usepackage{comment}
\usepackage{enumitem}

\newtheorem{tetel}{Tétel}[chapter]
\theoremstyle{definition}
\newtheorem{definicio}[tetel]{Definíció}
\theoremstyle{remark}
\newtheorem{megjegyzes}[tetel]{Megjegyzés}

\begin{document}
	\institute{Matematikai és Informatikai Intézet}
	\title{Informatikai eszközökkel támogatott sport és egészségfejlesztés}
	\author{Sipos Levente\\Szak: Programtervező informatikus BSc\\Specializáció: Szoftverfejlesztő informatikus}
	\supervisor{Dr. Király Roland\\beosztás}
	\city{Eger}
	\date{2022}
	\begin{titlepage}
		\maketitle
	\end{titlepage}
	
	\tableofcontents

	\addcontentsline{toc}{chapter}{Bevezetés}
	\addcontentsline{toc}{chapter}{Elméleti háttér}
	\addcontentsline{toc}{chapter}{Összegzés}



	\begin{comment}
		leírom hogy miről szól, (nem kötelező), 2 rész -> probléma felvetés, mi a motiváció, kontextus, előremutatva a dolgozatban hol miről fogsz beszélni.(1.fejezetben befogom mutatni a kódolást)
	\end{comment}
	\chapter*{Bevezetés}
	A tanulmányaim során sok olyan tárgyat tanulhattam amelyek segítettek belátást nyerni, hogy valójában melyik is az az irányágazat az informatikán belül, amely felkeltette az érdeklődésemet. Az utolsó félévekben tanulhattam robotikát a Robotika alapjai nevezetű tárgy következtében, amely közelebb vitt engem a gépközeli programozás világába. Továbbá C\# nyelvben elég biztos tudást szerezhettem a Szolgáltatás Orientált Programozás, Magasszintű programozási nyelvek I. és II. című tárgyakon.
	\par
	Szakdolgozatom témájául szerettem volna egy olyan szakdolgozatot készíteni, melynek a későbbiekben másoknak tudok segítséget nyújtani az informatikai szaktudásommal.
	\par
	A választott téma, mind az informatika mind, az egészségügy számára fontos kérdéseket tehet fel:
	\begin{itemize}
		\item  Mi jelenthet arra megoldást, ha egy adott korosztályba tartozó ember, nehézségekkel küzd a mindennapokban, a mozgását, illetve a mentális felfogását illetően? (Akár ez jelentheti az egyszerű mozgások nem megfelelő elvégzését, akár pedig az alap információk felfogásában való akadályozottságot is.)
		\item  Az informatikával tudunk-e az előbb említett kérdésre, olyan alkalmazást írni, amely ezeknek a fejlődését elősegítheti?
		\item Amennyiben tudunk ilyen alkalmazást írni, hogyan valósítsuk meg?
	\end{itemize}
	Alapvetően a következő fejezetekben azt szeretném részletezni, hogy milyen technológiákat használunk, és emellett milyen programozási nyelven készül a projekt. \\ 
	Továbbá, ki fogok térni azokra a rendszerekre is, amelyek hasonló céllal készültek el. Majd ezen projekteket hasonlóságait és különbségeit mérném össze, az elkészült projektünkkel. 
	
	\begin{comment}
	Bevezető2: Rendszer amit kidolgozott Somodi László, mi a szerepe, mi a célja, mit vállaltam én? -> frontend
	\end{comment}
	\chapter*{Bevezető 2}
	\par
	A témában jártas, és a "Mozgáskoordináció- és gyorsaságfejlesztő gyakorlatok óvodától a felnőtt korig" \cite{SLaszlo} című könyv írója, Somodi László, segített belátást nyerni az egészségügyi és a morális lényegességébe a projektnek. \\ Elmondása szerint a mozgásfejlesztés és az agyi kapacitás fejlesztése, kéz a kézben jár. Ezt a mozgásfejlesztést úgy érhetjük el, ha az adott személynek utasításokat adunk ki, hogy adott jelzésre (szín, hang, irány) és ezek kombinációjára, milyen mozgást kell végeznie. Továbbá, ez a módszer akár szellemi leépüléssel küzdő, szépkorúak számára is segítséget nyújthat, mivel a változatos mozgás, és a különböző ingerek ki- és be- kapcsolása növeli az agy alap működését.
	\par 
	A fentebb említettek automatizálására készül a projekt, amely különböző informatikai eszközökkel valósítja meg a színek, hangok, és nyilak megjelenítését, illetve érzékeltetését. 
	\begin{comment}
		Itt még lehetne kicsit beszélni a Somodi Lászlós dologokról.
	\end{comment}
\\
	Alapvetően, a projekten sok személy részt vett, a hardver lefejlesztésében és 
	összeszerelésében, Keresztes Péter Tanár úr. \\
	A backend és ezeknek a hardvereknek a mögöttes működtetését, valamint a Delphi és a C$\sharp$ nyelvek közötti kapcsolat megoldását, Nagy-Tóth Bence, barátom és szaktársam készítette el.
	\\
	Én ezeknek a hardvereknek a működtetéséhez a felületet írtam, amin keresztül lehet különféle módon, változatos ütemekben vezérelni a fentebb említett eszközöket. Ezt C$\sharp$ nyelven írtam ami a felhasználói felület írására kellően alkalmas.
	
	
	\begin{comment}
		1.fejezet: Milyen technológiákat használunk, milyen nyelv, architektúrák, kapcsolási rajz, interface, ui, meg ezek hogyan néznek ki.
	\end{comment}
	\chapter*{1.fejezet}
	\begin{comment}
		2.fejezet: Melyik részét csinálom én? Mit csináltunk, és hogyan ui-t kellett fejleszteni, milyen lépésekből állt, képernyőképekkel, eladjuk a munkánkat.
	\end{comment}
	\chapter*{2.fejezet}
	\begin{comment}
		3.fejezet(vége): kitekintés, milyen más rendszerek vannak még, és ahhoz képest ez mit tud, vannak hasonló szoftverek, vannak tréner szoftverek, de a mienk az spec képzésre alkalmas
		- szoftver és a hardvert hozzákellett fejleszteni, ez egy prototípus.
	\end{comment}
	\chapter*{3.fejezet}
	\begin{comment}
		4.fejezet: (future works) továbbiakban milyen kimenetele lesz, ebből szabadalom lesz, az ötletgazda szeretné ezt kiadni olyan módon hogy piaci termék legyen. 
		- Teljesen más technológia
	\end{comment}
	\chapter*{4.fejezet}
	\begin{comment}
		5.fejezet:  összefoglaló, melyik fejezetben miről beszéltünk, miket sikerült elérni(pl 1.fejezetben említetteket megcsináltam, iylenek), Mi az EREDMÉNY! miért fontos, 
		- miért jó ez? Működés közben videókat kéne csinálni. Érdemes lenne titkosítani.
	\end{comment}
	\chapter*{Összegzés}
	


	
	\verb*|#TODO|: Összefoglalás...
	\bibliographystyle{plain}
	\bibliography{references}
	\begin{comment}
		1.Alapötlet, Projekt célja, és hatásai az oktatásra/egézségügyre.
				Miért jó a projekt amit csinálunk?
				Somodi László által kapottak megemlítése
		2.A Winformos projekt
				milyen eszközöket lehet irányítani vele(mire jók)
				panelek külön-külön mit csinálnak
				hogyan kommunikál a Delphi-vel (Bence projektje, megemlítés)
		3. Alacsonyabbszintű komponensek/Imperatív és oop közti kommunikáció(Bence)
				Hangvezérlés
				Színek
				Nyilak
				4 || 8 eszközös megoldások
		4. Projekt tényleges használata, felmérések, és vélemények.
				Gondolatok és meglátások //akár lehet egyel előbb is
	\end{comment}
	% Aláírt, szkennelt nyilatkozat beillesztése a szakdolgozat végére
	%\includepdf{nyilatkozat.pdf}
\end{document}